\PassOptionsToPackage{dvipsnames}{xcolor}
\documentclass[10pt, xcolor={usenames, dvipsnames}]{beamer}

\usepackage[scale=2]{ccicons}
\usepackage{graphicx}
\usepackage{booktabs}
\usepackage{gensymb}
\usepackage{multimedia}
\usepackage{hyperref}
\usepackage{txfonts}
\usepackage{caption}
\usepackage{subcaption}

\usepackage[style=authoryear,backend=biber]{biblatex}
\renewcommand*{\nameyeardelim}{\addcomma\addspace}
\addbibresource{bib/abstract.bib}

% Beamer configuration
\usetheme[sectionpage=progressbar, numbering=counter, progressbar=frametitle]{metropolis}

\usepackage{pgfplots}
\usepackage{pgfplotsthemetol}
\usepackage{tikz}
\usetikzlibrary{positioning,arrows,decorations.pathmorphing,calc,patterns,decorations.markings,decorations.shapes,shapes.geometric}

\tikzset{paint/.style={ draw=#1, fill=#1 },
         decorate with/.style=
{decorate,decoration={shape backgrounds,shape=#1,shape size=1mm,shape sep=.5cm}}}

% Progressbar
\setbeamercolor{progress bar}{
    fg=TolLightGreen,
    bg=TolLightGreen!50!black!30
}
\makeatletter
    \setlength{\metropolis@titleseparator@linewidth}{2pt}
    \setlength{\metropolis@progressonsectionpage@linewidth}{2pt}
    \setlength{\metropolis@progressinheadfoot@linewidth}{2pt}
\makeatother

% Footer
\setbeamertemplate{frame footer}{Quentin Brateau, ENSTA Bretagne}

% Block fill
\metroset{block=fill}

\title{Torpedo-like AUV control in constrained environment}
\date{\today}
\author{Quentin Brateau}
\institute{ENSTA Bretagne}

\titlegraphic{
    \centering
    \begin{tabular}{lllll}
        \href{https://www.defense.gouv.fr/aid}{\includegraphics[height=0.6cm]{imgs/logo_aid}} &
        \href{https://www.gdr-robotique.org/}{\includegraphics[height=0.6cm]{imgs/logo_gdr}} &
        \href{https://www.ensta-bretagne.fr/fr/}{\includegraphics[height=0.6cm]{imgs/logo_ensta}} &
        \href{https://labsticc.fr/fr}{\includegraphics[height=0.6cm]{imgs/logo_labsticc}} &
        \href{https://www.ensta-bretagne.fr/robex/}{\includegraphics[height=0.6cm]{imgs/logo_robex}}
    \end{tabular}
}

\addtobeamertemplate{frametitle}{}{%
    \begin{tikzpicture}[remember picture,overlay]
    \node[anchor=north east,yshift=2pt] at (current page.north east) {\includegraphics[height=0.9cm]{imgs/logo_ensta_bw}};
    \end{tikzpicture}
}

\begin{document}

    \maketitle

    \begin{frame}{Table of contents}
        \setbeamertemplate{section in toc}[sections numbered]
        \tableofcontents[hideallsubsections]
    \end{frame}

    \section{Context}

        \begin{frame}{Introduction}{PhD}
            \centering
            \begin{minipage}[c]{0.58\textwidth}
                \begin{block}{Research laboratory}
                    \vspace{0.2cm}
                    \begin{itemize}
                        \item ENSTA Bretagne, UMR 6285, Lab-STICC
                    \end{itemize}
                \end{block}

                \begin{block}{Supervisiors}
                    \begin{itemize}
                        \item Luc Jaulin
                        \item Fabrice Lebars
                    \end{itemize}
                \end{block}

                \begin{block}{Funding}
                    \begin{itemize}
                        \item AID funding: Jean-Daniel Masson
                    \end{itemize}
                \end{block}
            \end{minipage}
            \hfill
            \begin{minipage}[c]{0.4\textwidth}
                \includegraphics[height=0.7\textheight, trim={24cm 0 16cm 0}, clip]{imgs/ensta.jpg}
            \end{minipage}
        \end{frame}

        \begin{frame}{Introduction}
            \begin{minipage}[t]{0.55\textwidth}
                \begin{block}{AUV}
                    \vspace{0.25cm}
                    \begin{itemize}
                        \item Control of torpedo-like AUV \\ 
                        \item Riptide's micro-uuv
                    \end{itemize}
                \end{block}
                \begin{block}{Environment}
                    \begin{itemize}
                        \item Constrained environment \\ 
                        \item Pool, harbor, ...
                    \end{itemize}
                \end{block}
                \begin{block}{Goals}
                    \begin{itemize}
                        \item Reactivity \\
                        \item Manoeuvrability
                    \end{itemize}
                \end{block}
            \end{minipage}
            \hfill
            \begin{minipage}[t]{0.4\textwidth}
                \begin{figure}[htb]
                    \includegraphics[width=\textwidth]{imgs/harbour.png}

                    \vspace{.1cm}

                    \includegraphics[width=\textwidth]{imgs/riptide.png}
                    \caption{Harbor and Riptide in the ENSTA Bretagne pool}
                \end{figure}
            \end{minipage}
        \end{frame}

    \section{Motivation}

        \begin{frame}{Obstacle avoidance}
            \begin{minipage}[b]{0.5\textwidth}
                \begin{block}{Wall avoidance}
                    \vspace{0.2cm}
                    \begin{itemize}
                        \item Sense wall \\
                        \item Determine normal vector $\mathbf{u}$ \\
                        \item Reorient AUV orthogonal to $\mathbf{u}$
                    \end{itemize}
                \end{block}
                \begin{block}{Control law}
                    \begin{itemize}
                        \item Without singularities \\
                        \item Based on reliable sensors \\
                        \item Independant of orientation \\
                        \item As fast as possible
                    \end{itemize}
                \end{block}
            \end{minipage}
            \hfill
            \begin{minipage}[b]{0.46\textwidth}
                \begin{figure}
                    \centering
                    \includegraphics[width=0.95\textwidth]{imgs/vodelee.png}
                    \caption{Vodelée Career - Belgium}
                \end{figure}
            \end{minipage}
        \end{frame}

        \begin{frame}{Classical control}
            \centering
            \begin{minipage}{0.45\textwidth}
                \centering
                \begin{figure}
                    \begin{tikzpicture}
                        \shade[ball color = gray!40, opacity = 0.4] (0,0) circle (2cm);
                        \draw[thick] (0,0) circle (2cm);
                        \onslide<2->{
                            \coordinate (R1) at (0.8,1.3);
                            \draw[thick,red,->,>=latex] (0,0) -- node[midway,above left] {$\mathbf{u}$} (R1); 
                        }
                        \onslide<3->{
                            \coordinate (R2) at (-0.2,1.15);
                            \draw[thick,ForestGreen,->,>=latex] (0,0) -- node[midway,above left] {$\mathbf{v}$} (R2); 
                        }
                        \onslide<4->{
                            \begin{scope}[rotate around={-30:(0,0)}]
                                \draw[thick] (-2,0) arc (180:360:2 and 0.6) coordinate[pos=0.2] (R3);
                                \draw[dashed] (2,0) arc (0:180:2 and 0.6);
                            \end{scope}
                            \draw[thick,RoyalBlue,->,>=latex] (0,0) -- node[midway,above] {$u_\bot$} (R3);
                            }
                        \onslide<5-> {
                            \path[thick,->,>=latex,RoyalPurple] (R2) edge[in=60,out=160] (R3);
                            \coordinate[below right=1.2mm and 1.2mm of R3] (R4);
                            % \draw[thick,->,>=latex,RoyalPurple] (R4) arc (-45:-225:2mm);
                        }
                    \end{tikzpicture}
                    \caption{Representation in $S^2$}
                \end{figure}
            \end{minipage}
            \begin{minipage}{0.45\textwidth}
                \centering
                \begin{figure}
                    \begin{tikzpicture}
                        \shade[ball color = gray!40, opacity = 0.4] (0,0) circle (2cm);
                        \draw[thick] (0,0) circle (2cm);
                        \onslide<2->{
                            \coordinate (R1) at (0.8,1.3);
                            \node[red] at (R1) {$\bullet$} node[red] at (R1) [below right] {$\mathbf{R}_u$};
                        }
                        \onslide<3->{
                            \coordinate (R2) at (-0.2,1.15);
                            \node[ForestGreen] at (R2) {$\bullet$} node[ForestGreen] at (R2) [above] {$\mathbf{R}_v$};
                        }
                        \onslide<4->{
                            \begin{scope}[rotate around={-30:(0,0)}]
                                \draw[thick] (-2,0) arc (180:360:2 and 0.6) coordinate[pos=0.2] (R3);
                                \draw[dashed] (2,0) arc (0:180:2 and 0.6);
                            \end{scope}
                            \node[thick,RoyalBlue] at (R3) {$\bullet$} node[RoyalBlue] at (R3) [below] {$\mathbf{R}_{u_\bot}$};
                        }
                        \onslide<5->{
                            \path[thick,->,>=latex,RoyalPurple] (R2) edge[in=60,out=160] node[midway,below] {$R_w$} (R3);
                        }
                    \end{tikzpicture}
                    \caption{Representation in $SO(3)$}
                \end{figure}
            \end{minipage}
        \end{frame}

        \begin{frame}{Classical control}
            \centering
            \begin{minipage}{0.75\textwidth}
                \begin{block}{Strengths}
                    \vspace{0.2cm}
                    \begin{itemize}
                        \item Often already implemented for other purposes
                        \item AUV orientation fully controlled
                    \end{itemize}
                \end{block}
                \begin{block}{Weaknesses}
                    \begin{itemize}
                        \item AUV angles have to be known
                        \item Not the quickest reorientation
                        \item Set all angles
                    \end{itemize}
                \end{block}
            \end{minipage}
        \end{frame}

        \begin{frame}{Proposed approach}
            \centering
            \begin{minipage}{0.45\textwidth}
                \centering
                \begin{figure}
                    \begin{tikzpicture}
                        \shade[ball color = gray!40, opacity = 0.4] (0,0) circle (2cm);
                        \draw[thick] (0,0) circle (2cm);
                        \onslide<2->{
                            \coordinate (R1) at (0.8,1.3);
                            \draw[thick,red,->,>=latex] (0,0) -- node[midway,above left] {$\mathbf{u}$} (R1); 
                        }
                        \onslide<3->{
                            \coordinate (R2) at (-0.2,1.15);
                            \draw[thick,ForestGreen,->,>=latex] (0,0) -- node[midway,above left] {$\mathbf{v}$} (R2); 
                        }
                        \onslide<4->{
                            \begin{scope}[rotate around={-30:(0,0)}]
                                \draw[thick] (-2,0) arc (180:360:2 and 0.6) coordinate[pos=0.36] (R3) coordinate[pos=0.61] (R4);
                                \draw[dashed] (2,0) arc (0:180:2 and 0.6);
                            \end{scope}
                            \draw[thick,dotted] (R1) edge[in=30,out=175] (R2);
                            \draw[thick,RoyalBlue,->,>=latex] (0,0) -- node[midway,above] {$\mathbf{v_\bot}$} (R3);
                        }
                        \onslide<4>{
                            \draw[thick,dotted] (R2) edge[in=70,out=210] (R3);
                        }
                        \onslide<5->{
                            \path[thick,->,>=latex,RoyalPurple] (R2) edge[in=70,out=210] (R3);
                            \draw[thick,RoyalPurple,->,>=latex] (0,0) -- node[midway,right] {$\mathbf{w}$} (R4);
                        }
                    \end{tikzpicture}
                    \caption{Representation in $S^2$}
                \end{figure}
            \end{minipage}
            \begin{minipage}{0.45\textwidth}
                \centering
                \begin{figure}
                    \begin{tikzpicture}
                        \shade[ball color = gray!40, opacity = 0.4] (0,0) circle (2cm);
                        \draw[thick] (0,0) circle (2cm);
                        \onslide<2->{
                            \coordinate (R1) at (0.8,1.3);
                            \node[red] at (R1) {$\bullet$} node[red] at (R1) [below right] {$\mathbf{R}_u$};
                        }
                        \onslide<3->{
                            \coordinate (R2) at (-0.2,1.15);
                            \node[ForestGreen] at (R2) {$\bullet$} node[ForestGreen] at (R2) [above left] {$\mathbf{R}_v$};
                        }
                        \onslide<4->{
                            \begin{scope}[rotate around={-30:(0,0)}]
                                \draw[thick] (-2,0) arc (180:360:2 and 0.6) coordinate[pos=0.36] (R3);
                                \draw[dashed] (2,0) arc (0:180:2 and 0.6);
                            \end{scope}
                            \node[thick,RoyalBlue] at (R3) {$\bullet$} node[RoyalBlue] at (R3) [below left] {$\mathbf{R}_{v_\bot}$};
                        }
                        \onslide<5->{
                            \path[thick,->,>=latex,RoyalPurple] (R2) edge[in=70,out=210] node[midway, left] {$R_w$} (R3);
                        }
                    \end{tikzpicture}
                    \caption{Representation in $SO(3)$}
                \end{figure}
            \end{minipage}
        \end{frame}

        \begin{frame}{Proposed approach}
            \centering
            \begin{minipage}{0.75\textwidth}
                \begin{block}{Strengths}
                    \vspace{0.2cm}
                    \begin{itemize}
                        \item Set only necessary angles
                        \item Other angles don't need to be known
                        \item Quickest reorientation
                    \end{itemize}
                \end{block}
                \begin{block}{Weaknesses}
                    \begin{itemize}
                        \item Chosen direction not controlled
                    \end{itemize}
                \end{block}
            \end{minipage}
        \end{frame}

    \section{Orthogonal control}

        \begin{frame}{Determine $\mathbf{v_\bot}$}
            \begin{minipage}{0.4\textwidth}
                \centering
                \begin{figure}
                    \begin{tikzpicture}
                        \shade[ball color = gray!40, opacity = 0.4] (0,0) circle (2cm);
                        \draw[thick] (0,0) circle (2cm);
                        \coordinate (R1) at (0.8,1.3);
                        \draw[thick,red,->,>=latex] (0,0) -- node[midway,above left] {$\mathbf{u}$} (R1);
                        \begin{scope}[rotate around={-30:(0,0)}]
                            \draw[thick] (-2,0) arc (180:360:2 and 0.6) coordinate[pos=0.36] (R3);
                            \draw[dashed] (2,0) arc (0:180:2 and 0.6);
                        \end{scope}
                        \coordinate (R2) at (-0.2,1.15);
                        
                        \draw[thick,ForestGreen,->,>=latex] (0,0) -- node[midway,left]  {$\mathbf{v}$} (R2);
                        \onslide<2> {
                            \coordinate (vu) at ($(0,0)!0.78!(R1)$);
                            \draw[thick,dotted] (vu) -- (R2);
                            \draw[thick,RoyalPurple,->,>=latex] (0,0) -- node[midway,right] {$\langle \mathbf{u}, \mathbf{v}\rangle \cdot \mathbf{u}$} (vu);
                        }
                        \onslide<3> {
                            \coordinate (vv) at ($(0,0)!0.72!(R3)$);
                            \draw[thick,RoyalPurple,->,>=latex] (vv) -- node[midway,left] {$\langle \mathbf{u}, \mathbf{v}\rangle \cdot \mathbf{u}$} (R2);
                            \draw[thick,RoyalBlue,->,>=latex] (0,0) -- node[midway,below right] {$\mathbf{v} - \langle \mathbf{u}, \mathbf{v}\rangle \cdot \mathbf{u}$} (vv);
                            }
                        \onslide<4> {
                            \draw[thick,RoyalBlue,->,>=latex] (0,0) -- node[midway,below right] {$\mathbf{v_\bot}$} (R3);
                        }
                    \end{tikzpicture}
                    \caption{Representation in $S^2$}
                \end{figure}
            \end{minipage}
            \hfill
            \begin{minipage}{0.55\textwidth}
                \begin{block}{Determine $\mathbf{v_\bot}$}
                    \begin{equation}
                        \mathbf{\color{RoyalBlue}{v_\bot}} = \frac{\mathbf{\color{ForestGreen}{v}} - \langle \mathbf{\color{red}{u}}, \mathbf{\color{ForestGreen}{v}}\rangle \cdot \mathbf{\color{red}{u}}}{||\mathbf{\color{ForestGreen}{v}} - \langle \mathbf{\color{red}{u}}, \mathbf{\color{ForestGreen}{v}}\rangle \cdot \mathbf{\color{red}{u}}||}
                    \end{equation}
                \end{block}
                
            \end{minipage}
        \end{frame}

        \begin{frame}{Determine $\mathbf{R_w}$ - Classical way}
            \begin{minipage}{0.4\textwidth}
                \centering
                \begin{figure}
                    \begin{tikzpicture}
                        \shade[ball color = gray!40, opacity = 0.4] (0,0) circle (2cm);
                        \draw[thick] (0,0) circle (2cm);
                        \coordinate (R1) at (0.8,1.3);
                        \draw[thick,red,->,>=latex] (0,0) -- node[midway,above left] {$\mathbf{u}$} (R1);
                        \draw[thick,RoyalBlue,->,>=latex] (0,0) -- node[midway,above] {$\mathbf{v_\bot}$} (R3);
                        \begin{scope}[rotate around={-30:(0,0)}]
                            \draw[thick] (-2,0) arc (180:360:2 and 0.6) coordinate[pos=0.36] (R3) coordinate[pos=0.61] (R4);
                            \draw[dashed] (2,0) arc (0:180:2 and 0.6);
                        \end{scope}
                        
                        \onslide<1>{
                            \coordinate (R2) at (-0.2,1.15);
                            \draw[thick,ForestGreen,->,>=latex] (0,0) -- node[midway,above left] {$\mathbf{v}$} (R2); 
                            \draw[thick,dotted] (R1) edge[in=30,out=175] (R2);
                            \path[thick,->,>=latex,RoyalPurple] (R2) edge[in=70,out=210] node[midway, left] {$\alpha$} (R3);
                            \draw[thick,RoyalPurple,->,>=latex] (0,0) -- node[midway,right] {$\mathbf{w}$} (R4);
                        }
                        \onslide<2->{
                            \draw[thick,ForestGreen,->,>=latex] (0,0) -- node[midway,below right] {$\mathbf{v}$} (R3);
                        }
                    \end{tikzpicture}
                    \caption{Representation in $S^2$}
                \end{figure}
            \end{minipage}
            \hfill
            \begin{minipage}{0.55\textwidth}
                \begin{block}{Determine $R_w$}
                    \begin{equation}
                        \begin{split}
                            \mathbf{\color{RoyalPurple}{w}} &= \mathbf{\color{ForestGreen}{v}} \wedge \mathbf{\color{RoyalBlue}{v_\bot}} \\
                            \color{RoyalPurple}{\alpha} &= arccos(\mathbf{||\color{RoyalPurple}{w}}||) \\
                            \mathbf{R_{\color{RoyalPurple}{w}}} &= exp\left(
                                {\color{RoyalPurple}{\alpha}}
                                \frac{
                                    \mathbf{\color{RoyalPurple}{w}}
                                }{
                                    \color<2>{red}{||\mathbf{\color<1>{RoyalPurple}{w}}}||
                                }
                                t\right)
                        \end{split}
                    \end{equation}
                \end{block}
                \begin{block}<2->{Issue}
                    \vspace*{0.25cm}
                    \begin{itemize}
                        \item If $\mathbf{\color{ForestGreen}{v}} = \mathbf{\color{RoyalBlue}{v_\bot}}$, $||\mathbf{\color{RoyalPurple}{w}}|| = 0$
                    \end{itemize}
                \end{block}
            \end{minipage}
        \end{frame}

        \begin{frame}{Determine $\mathbf{R_w}$ - Codesido flavor}
            \begin{minipage}{0.4\textwidth}
                \centering
                \begin{figure}
                    \begin{tikzpicture}
                        \shade[ball color = gray!40, opacity = 0.4] (0,0) circle (2cm);
                        \draw[thick] (0,0) circle (2cm);
                        \coordinate (R1) at (0.8,1.3);
                        \draw[thick,red,->,>=latex] (0,0) -- node[midway,above left] {$\mathbf{u}$} (R1);
                        \draw[thick,RoyalBlue,->,>=latex] (0,0) -- node[midway,above] {$\mathbf{v_\bot}$} (R3);
                        \begin{scope}[rotate around={-30:(0,0)}]
                            \draw[thick] (-2,0) arc (180:360:2 and 0.6) coordinate[pos=0.36] (R3) coordinate[pos=0.61] (R4);
                            \draw[dashed] (2,0) arc (0:180:2 and 0.6)  coordinate[pos=0.36] (R5);
                        \end{scope}
                        
                        \onslide<1>{
                            \coordinate (R2) at (-0.2,1.15);
                            \draw[thick,ForestGreen,->,>=latex] (0,0) -- node[midway,above left] {$\mathbf{v}$} (R2); 
                            \draw[thick,dotted] (R1) edge[in=30,out=175] (R2);
                            \path[thick,->,>=latex,RoyalPurple] (R2) edge[in=70,out=210] (R3);
                            \draw[thick,RoyalPurple,->,>=latex] (0,0) -- node[midway,right] {$\mathbf{w}$} (R4);
                        }

                        \onslide<2>{
                            \draw[thick,ForestGreen,->,>=latex] (0,0) -- node[midway,below right] {$\mathbf{v}$} (R3);
                        }

                        \onslide<3>{
                            \draw[thick,ForestGreen,->,>=latex] (0,0) -- node[midway,below right] {$\mathbf{v}$} (R5);
                        }
                    \end{tikzpicture}
                    \caption{Representation in $S^2$}
                \end{figure}
            \end{minipage}
            \hfill
            \begin{minipage}{0.55\textwidth}
                \begin{block}{Codesido formula \footcite{Codesido}}
                    \begin{equation}
                        \begin{array}{rcl}
                            \mathbf{K}_{\mathbf{\color{ForestGreen}{v}}}^{\mathbf{\color{RoyalBlue}{v_\bot}}} & = & \mathbf{\color{RoyalBlue}{v_\bot}} \mathbf{\color{ForestGreen}{v}}^T - \mathbf{\color{ForestGreen}{v}}\mathbf{\color{RoyalBlue}{v_\bot}}^T \\
                            \mathbf{R}_{\mathbf{\color{ForestGreen}{v}}}^{\mathbf{\color{RoyalBlue}{v_\bot}}} & = & \mathbf{I_3} + \mathbf{K}_{\mathbf{\color{ForestGreen}{v}}}^{\mathbf{\color{RoyalBlue}{v_\bot}}} + \frac{1}{1 + \color<3>{red}{\langle \mathbf{\color<-2>{ForestGreen}{v}}, \mathbf{\color<-2>{RoyalBlue}{v_\bot}}\rangle}} (\mathbf{K}_{\mathbf{\color{ForestGreen}{v}}}^{\mathbf{\color{RoyalBlue}{v_\bot}}})^2 \\
                        \end{array}
                    \end{equation}
                \end{block}
                % \begin{block}<2->{Angular velocity w}
                %     \begin{equation}
                %         \mathbf{\color{RoyalPurple}{w}} = \frac{1}{t} Log(\mathbf{R}_{\mathbf{\color{ForestGreen}{v}}}^{\mathbf{\color{RoyalBlue}{v_\bot}}})
                %     \end{equation}
                % \end{block}
                \begin{block}<2->{No singularities}
                    \vspace{.2cm}
                    \begin{itemize}
                        \item If $\mathbf{\color{ForestGreen}{v}} = \mathbf{\color{RoyalBlue}{v_\bot}}$, $\mathbf{R_{{\color{ForestGreen}{v}}}^{{\color{RoyalBlue}{v_\bot}}}} = \mathbf{I_3}$ 
                        \item<3> Singularity when $\langle\mathbf{\color{ForestGreen}{v}}, \mathbf{\color{RoyalBlue}{v_\bot}}\rangle = -1$
                    \end{itemize}
                \end{block}
            \end{minipage}
        \end{frame}

        \begin{frame}{Limitations}
            \begin{minipage}{0.4\textwidth}
                \centering
                \begin{figure}
                    \begin{tikzpicture}
                        \shade[ball color = gray!40, opacity = 0.4] (0,0) circle (2cm);
                        \draw[thick] (0,0) circle (2cm);
                        \coordinate (R1) at (0.8,1.3);
                        \draw[thick,red,->,>=latex] (0,0) -- node[midway,above left] {$\mathbf{u}$} (R1);
                        \begin{scope}[rotate around={-30:(0,0)}]
                            \draw[thick] (-2,0) arc (180:360:2 and 0.6);
                            \draw[dashed] (2,0) arc (0:180:2 and 0.6);
                        \end{scope}
                        \draw[thick,ForestGreen,->,>=latex] (0,0) -- node[midway,right] {$\mathbf{v}$} (R1); 
                    \end{tikzpicture}
                    \caption{Representation in $S^2$}
                \end{figure}
            \end{minipage}
            \hfill
            \begin{minipage}{0.55\textwidth}
                \begin{block}{Limitation}
                    Physical singularity when $\mathbf{\color{red}{u}} = \mathbf{\color{ForestGreen}{v}}$, as $\mathbf{\color{RoyalBlue}{v_\bot}}$ is undefined
                \end{block}
            \end{minipage}
        \end{frame}

        \begin{frame}{Controller implementation}
            \centering
            \begin{minipage}{0.8\textwidth}
                \centering
                \begin{figure}
                    \begin{tikzpicture}[
                        input/.style={->,>=latex,thick,decorate,
                        decoration={snake,amplitude=.4mm,segment length=2mm,post length=2mm}},
                        block/.style={draw,font=\small,thick,
                            minimum width={2cm},minimum height={1.5cm}}]

                        \node[block,circle,align=center] (n1) {Orthogonal\\controller};
                        \node[block,circle,right=of n1] (n2) {Controller};
                        \node[block,rectangle,right=of n2] (n3) {AUV};

                        \draw[thick,->,>=latex,RoyalBlue] (n1) to node[midway,above] {$\mathbf{w}$} (n2);
                        
                        \foreach \i/\a/\s in {0/37.5/-2.4,1/12.5/-3,2/-12.5/-3,3/-37.5/-2.5} {
                            \draw[thick,->,>=latex,ForestGreen] (n2.\a) to node[near end,xshift=\s*1mm,yshift=1.5mm] {$u_\i$} ([yshift=0.6cm -\i * 0.4 cm]n3.west);
                        }
                        
                        % inputs
                        \coordinate[above=of n1.100] (u);
                        \coordinate[above=of n1.80] (v);
                        \draw[input,red] (u) -- node[midway,left,red] {$\mathbf{u}$} (n1.100);
                        \draw[input,ForestGreen] (v) -- node[midway,right,ForestGreen] {$\mathbf{v}$} (n1.80);
                    \end{tikzpicture}
                    \caption{Block diagram}
                \end{figure}
            \end{minipage}
            \hfill
        \end{frame}
    
    \section{Application}

        \begin{frame}{2D Wall avoidance - Classical way}
            \begin{minipage}{0.4\textwidth}
                \begin{figure}
                    \begin{tikzpicture}
                        \draw (0,-2) --++(90:4cm);
                        \fill[pattern=north west lines] (0,-2) rectangle ++(-.3,4);

                        \onslide<2-> {
                            \draw[thick,red,->,>=latex] (.25,0) -- ++(0,1) node[midway,right] {$\mathbf{u}$};
                        }
                        
                        \onslide<3-> {
                            \begin{scope}[xshift=3cm,yshift=.5cm]
                                \coordinate (c1) at (0,0);
                                \path[rotate=155,thick,draw] (-0.2,-0.25) -- (-0.2,0.25) -- (0.5,0) coordinate (n1) -- cycle;
                            \end{scope}
                            \path[draw,thick,RoyalBlue] (n1) arc (245:180:1.2cm) -- ++(0,.15);
                            \draw[rotate=155,thick,ForestGreen,->,>=latex] (c1) -- +(0:1cm) node[midway,above] {$\mathbf{v_1}$};
                        }

                        \onslide<4-> {
                            \begin{scope}[xshift=3cm,yshift=-.5cm]
                                \coordinate (c2) at (0,0);
                                \path[rotate=205,thick,draw] (-0.2,-0.25) -- (-0.2,0.25) -- (0.5,0) coordinate (n2) -- cycle;
                            \end{scope}
                            \path[draw,thick,RoyalBlue] (n2) arc (-65:-180:1.2cm) -- ++(0,1.6);
                            \draw[rotate=205,thick,ForestGreen,->,>=latex] (c2) -- +(0:1cm) node[midway,above] {$\mathbf{v_2}$};
                        }
                    \end{tikzpicture}
                    \caption{Wall avoidance - Classical control}
                \end{figure}
            \end{minipage}
            \hfill
            \begin{minipage}{0.55\textwidth}
                \begin{figure}
                    \begin{tikzpicture}[decoration=triangles]
                        \foreach \n/\a in {n1/90, n2/270} {
                            \coordinate (\n) at (\a:1.5cm);
                        }

                        \path[draw] (n2) arc (-90:90:1.5cm);
                        \path[draw] (n2) arc (270:90:1.5cm);

                        \onslide<2-> {
                            \path[draw,thick,RoyalBlue] (n2) arc (-90:90:1.5cm);
                            \path[draw,thick,RoyalBlue] (n2) arc (270:90:1.5cm);
                        }

                        \onslide<3-> {
                            \path[draw,thick,RoyalBlue,postaction={draw,decorate with=dart,paint=RoyalBlue}] (n2) arc (-90:90:1.5cm);
                            \path[draw,thick,RoyalBlue,postaction={draw,decorate with=dart,paint=RoyalBlue}] (n2) arc (270:90:1.5cm);
                            \draw[thick,ForestGreen,->,>=latex] (0,0) -- (155:1.5) node[midway,above] {$\mathbf{v_1}$};
                        }

                        \onslide<2-> {
                            \filldraw[RoyalBlue] (n1) circle (3pt) node[above] {control};
                            \filldraw[red] (n2) circle (3pt) node[below] {singularity};
                        }

                        \onslide<4-> {
                            \draw[thick,ForestGreen,->,>=latex] (0,0) -- (205:1.5) node[midway,below] {$\mathbf{v_2}$};
                        }

                        \filldraw (0,0) circle (1pt);
                    \end{tikzpicture}
                    \caption{Representation in $S^1$}
                \end{figure}
            \end{minipage}
        \end{frame}

        \begin{frame}{2D Wall avoidance - Orthogonal control}
            \begin{minipage}{0.4\textwidth}
                \begin{figure}
                    \begin{tikzpicture}
                        \draw (0,-2) --++(90:4cm);
                        \fill[pattern=north west lines] (0,-2) rectangle ++(-.3,4);

                        \onslide<2-> {
                            \draw[thick,red,->,>=latex] (0,0) -- ++(1,0) node[midway,above] {$\mathbf{u}$};
                        }
                        
                        \onslide<3-> {
                            \begin{scope}[xshift=3cm,yshift=.5cm]
                                \coordinate (c1) at (0,0);
                                \path[rotate=155,thick,draw] (-0.2,-0.25) -- (-0.2,0.25) -- (0.5,0) coordinate (n1) -- cycle;
                            \end{scope}
                            \path[draw,thick,RoyalBlue] (n1) arc (245:180:1.2cm) -- ++(0,.15);
                            \draw[rotate=155,thick,ForestGreen,->,>=latex] (c1) -- +(0:1cm) node[midway,above] {$\mathbf{v_1}$};
                        }
                        
                        \onslide<4-> {
                            \begin{scope}[xshift=3cm,yshift=-.5cm]
                                \coordinate (c2) at (0,0);
                                \path[rotate=205,thick,draw] (-0.2,-0.25) -- (-0.2,0.25) -- (0.5,0) coordinate (n2) -- cycle;
                            \end{scope}
                            \path[draw,thick,RoyalPurple] (n2) arc (115:180:1.2cm) -- ++(0,-.15);
                            \draw[rotate=205,thick,ForestGreen,->,>=latex] (c2) -- +(0:1cm) node[midway,above] {$\mathbf{v_2}$};
                        }
                    \end{tikzpicture}
                    \caption{Wall avoidance - Proposed approach}
                \end{figure}
            \end{minipage}
            \hfill
            \begin{minipage}{0.55\textwidth}
                \begin{figure}
                    \begin{tikzpicture}[decoration=triangles]

                        \foreach \n/\a in {n1/0, n2/90, n3/180, n4/270} {
                            \coordinate (\n) at (\a:1.5cm);
                        }

                        \path[draw] (n1.north) arc (0:90:1.5cm);
                        \path[draw] (n3.north) arc (180:90:1.5cm);
                        \path[draw] (n3.south) arc (180:270:1.5cm);
                        \path[draw] (n1.south) arc (360:270:1.5cm);

                        \onslide<2-> {
                            \path[draw,thick,RoyalBlue] (n1.north) arc (0:90:1.5cm);
                            \path[draw,thick,RoyalBlue] (n3.north) arc (180:90:1.5cm);
                            \path[draw,thick,RoyalPurple] (n3.south) arc (180:270:1.5cm);
                            \path[draw,thick,RoyalPurple] (n1.south) arc (360:270:1.5cm);
                        }

                        \onslide<3-> {
                            \path[draw,thick,RoyalBlue,postaction={draw,decorate with=dart,paint=RoyalBlue}] (n1.north) arc (0:90:1.5cm);
                            \path[draw,thick,RoyalBlue,postaction={draw,decorate with=dart,paint=RoyalBlue}] (n3.north) arc (180:90:1.5cm);
                        }

                        \onslide<4-> {
                            \path[draw,thick,RoyalPurple,postaction={draw,decorate with=dart,paint=RoyalPurple}] (n3.south) arc (180:270:1.5cm);
                            \path[draw,thick,RoyalPurple,postaction={draw,decorate with=dart,paint=RoyalPurple}] (n1.south) arc (360:270:1.5cm);
                        }

                        \onslide<2-> {
                            \foreach \n/\c in {n1/red, n2/RoyalBlue, n3/red, n4/RoyalPurple} {
                                \filldraw[\c] (\n) circle (3pt);
                            }
                            \filldraw[RoyalBlue] (n2) circle (3pt) node[above] {control};
                            \filldraw[red] (n1) circle (3pt) node[right] {singularity};
                            \filldraw[red] (n3) circle (3pt) node[left] {singularity};
                            \filldraw[RoyalPurple] (n4) circle (3pt) node[below] {control};
                        }

                        \onslide<3-> {
                            \draw[thick,ForestGreen,->,>=latex] (0,0) -- (155:1.5) node[midway,above] {$\mathbf{v_1}$};
                        }

                        \onslide<4-> {
                            \draw[thick,ForestGreen,->,>=latex] (0,0) -- (205:1.5) node[midway,below] {$\mathbf{v_2}$};
                        }
                        \filldraw (0,0) circle (1pt);
                    \end{tikzpicture}
                    \caption{Representation in $S^1$}
                \end{figure}
            \end{minipage}
        \end{frame}

        \begin{frame}{2D Pipe following - Classical way}
            \begin{minipage}{0.4\textwidth}
                \begin{figure}
                    \begin{tikzpicture}
                        \begin{scope}[rotate=35]
                            \def\r{.2};
                            \def\h{4};
                            \draw[fill=gray!60]  (-\h/2,\r) to[out=0,in=60]  (-\h/2,0) to[out=120,in=180] cycle;
                            \draw[fill=gray!30] (-\h/2,\r) to[out=0,in=60] (-\h/2,0) to[out=-120,in=180] (-\h/2,-\r) -- (\h/2,-\r) to[out=180,in=-120] (\h/2,0) to[out=60,in=0] (\h/2,\r) -- cycle;
                            \draw[fill=gray!60]  (\h/2,-\r) to[out=180,in=-120]  (\h/2,0) to[out=-60,in=0] cycle;
                        \end{scope}

                        \onslide<2-> {
                            \draw[thick,red,->,>=latex] (0,0) -- +(35:1cm) node[midway,xshift=-.2cm,yshift=.25cm] {$\mathbf{u}$};
                        }
                        \onslide<3-> {
                            \begin{scope}[xshift=-2cm,yshift=.5cm]
                                \coordinate (c2) at (0,0);
                                \path[rotate=-25,thick,draw] (-0.2,-0.25) -- (-0.2,0.25) -- (0.5,0) coordinate (n2) -- cycle;
                                \path[rotate=-25,draw,thick,RoyalBlue] (n2) arc (-90:-30:1.2cm) -- ++(60:2);
                                \draw[rotate=-25,thick,ForestGreen,->,>=latex] (c2) -- +(0:1cm) node[midway,above] {$\mathbf{v_1}$};
                            \end{scope}
                        }

                        \onslide<4-> {
                            \begin{scope}[xshift=2cm,yshift=-.5cm]
                                \coordinate (c1) at (0,0);
                                \path[rotate=155,thick,draw] (-0.2,-0.25) -- (-0.2,0.25) -- (0.5,0) coordinate (n1) -- cycle;
                                \path[rotate=155, draw,thick,RoyalBlue] (n1) arc (90:-25:1.2cm) -- ++(-115:0);
                                \draw[rotate=155,thick,ForestGreen,->,>=latex] (c1) -- +(0:1cm) node[midway,above] {$\mathbf{v_2}$};
                            \end{scope}
                        }
                    \end{tikzpicture}
                    \caption{Pipe following - Proposed approach}
                \end{figure}
            \end{minipage}
            \hfill
            \begin{minipage}{0.55\textwidth}
                \begin{figure}
                    \begin{tikzpicture}[decoration=triangles,rotate=-65]
                        \foreach \n/\a in {n1/90, n2/270} {
                            \coordinate (\n) at (\a:1.5cm);
                        }
                        \path[draw] (n2) arc (-90:90:1.5cm);
                        \path[draw] (n2) arc (270:90:1.5cm);

                        \onslide<2-> {
                            \path[draw,thick,RoyalBlue] (n2) arc (-90:90:1.5cm);
                            \path[draw,thick,RoyalBlue] (n2) arc (270:90:1.5cm);
                        }

                        \onslide<3-> {
                            \path[draw,thick,RoyalBlue,postaction={draw,decorate with=dart,paint=RoyalBlue}] (n2) arc (-90:90:1.5cm);
                        }

                        \onslide<4-> {
                            \path[draw,thick,RoyalBlue,postaction={draw,decorate with=dart,paint=RoyalBlue}] (n2) arc (270:90:1.5cm);
                        }
                        
                        \onslide<2-> {
                            \filldraw[RoyalBlue] (n1) circle (3pt) node[yshift=.3cm,right] {control};
                            \filldraw[red] (n2) circle (3pt) node[yshift=-.3cm,left] {singularity};
                        }

                        \onslide<3-> {
                            \draw[thick,ForestGreen,->,>=latex] (0,0) -- (40:1.5) node[midway,below] {$\mathbf{v_1}$};
                        }

                        \onslide<4-> {
                            \draw[thick,ForestGreen,->,>=latex] (0,0) -- (210:1.5) node[midway,above] {$\mathbf{v_2}$};
                        }

                        \filldraw (0,0) circle (1pt);
                    \end{tikzpicture}
                    \caption{Representation in $S^1$}
                \end{figure}
            \end{minipage}
        \end{frame}

        \begin{frame}{2D Wall avoidance - Orthogonal control}
            \begin{minipage}{0.4\textwidth}
                \begin{figure}
                    \begin{tikzpicture}
                        \begin{scope}[rotate=35]
                            \def\r{.2};
                            \def\h{4};
                            \draw[fill=gray!60]  (-\h/2,\r) to[out=0,in=60]  (-\h/2,0) to[out=120,in=180] cycle;
                            \draw[fill=gray!30] (-\h/2,\r) to[out=0,in=60] (-\h/2,0) to[out=-120,in=180] (-\h/2,-\r) -- (\h/2,-\r) to[out=180,in=-120] (\h/2,0) to[out=60,in=0] (\h/2,\r) -- cycle;
                            \draw[fill=gray!60]  (\h/2,-\r) to[out=180,in=-120]  (\h/2,0) to[out=-60,in=0] cycle;
                        \end{scope}

                        \onslide<2-> {
                            \draw[thick,red,->,>=latex] (0,0) -- +(125:1cm) node[midway,xshift=.1cm,yshift=.35cm] {$\mathbf{u}$};
                        }

                        \onslide<3-> {
                            \begin{scope}[xshift=-2cm,yshift=.5cm]
                                \coordinate (c2) at (0,0);
                                \path[rotate=-25,thick,draw] (-0.2,-0.25) -- (-0.2,0.25) -- (0.5,0) coordinate (n2) -- cycle;
                                \path[rotate=-25,draw,thick,RoyalBlue] (n2) arc (-90:-30:1.2cm) -- ++(60:2);
                                \draw[rotate=-25,thick,ForestGreen,->,>=latex] (c2) -- +(0:1cm) node[midway,above] {$\mathbf{v_1}$};
                            \end{scope}
                        }
                        
                        \onslide<4-> {
                            \begin{scope}[xshift=2cm,yshift=-.5cm]
                                \coordinate (c1) at (0,0);
                                \path[rotate=155,thick,draw] (-0.2,-0.25) -- (-0.2,0.25) -- (0.5,0) coordinate (n1) -- cycle;
                                \path[rotate=155, draw,thick,RoyalPurple] (n1) arc (-90:-30:1.2cm) -- ++(60:2);
                                \draw[rotate=155,thick,ForestGreen,->,>=latex] (c1) -- +(0:1cm) node[midway,above] {$\mathbf{v_2}$};
                            \end{scope}
                        }
                    \end{tikzpicture}
                    \caption{Pipe following - Proposed approach}
                \end{figure}
            \end{minipage}
            \hfill
            \begin{minipage}{0.55\textwidth}
                \begin{figure}
                    \begin{tikzpicture}[decoration=triangles,rotate=-35]
                        \foreach \n/\a in {n1/0, n2/90, n3/180, n4/270} {
                            \coordinate (\n) at (\a:1.5cm);
                        }
                        \path[draw] (n1.north) arc (0:90:1.5cm);
                        \path[draw] (n3.north) arc (180:90:1.5cm);
                        \path[draw] (n3.south) arc (180:270:1.5cm);
                        \path[draw] (n1.south) arc (360:270:1.5cm);

                        \onslide<2-> {
                            \path[draw,thick,RoyalBlue] (n1.north) arc (0:90:1.5cm);
                            \path[draw,thick,RoyalBlue] (n3.north) arc (180:90:1.5cm);
                            \path[draw,thick,RoyalPurple] (n3.south) arc (180:270:1.5cm);
                            \path[draw,thick,RoyalPurple] (n1.south) arc (360:270:1.5cm);
                        }
                        
                        \onslide<3-> {
                            \path[draw,thick,RoyalBlue,postaction={draw,decorate with=dart,paint=RoyalBlue}] (n1.north) arc (0:90:1.5cm);
                            \path[draw,thick,RoyalBlue,postaction={draw,decorate with=dart,paint=RoyalBlue}] (n3.north) arc (180:90:1.5cm);
                        }

                        \onslide<4-> {
                            \path[draw,thick,RoyalPurple,postaction={draw,decorate with=dart,paint=RoyalPurple}] (n3.south) arc (180:270:1.5cm);
                            \path[draw,thick,RoyalPurple,postaction={draw,decorate with=dart,paint=RoyalPurple}] (n1.south) arc (360:270:1.5cm);
                        }

                        \onslide<2-> {
                            \foreach \n/\c in {n1/red, n2/RoyalBlue, n3/red, n4/RoyalPurple} {
                                \filldraw[\c] (\n) circle (3pt);
                            }
                            \filldraw[RoyalBlue] (n2) circle (3pt) node[yshift=.3cm,right] {control};
                            \filldraw[red] (n1) circle (3pt) node[right,xshift=.1cm] {singularity};
                            \filldraw[red] (n3) circle (3pt) node[left,xshift=-.1cm] {singularity};
                            \filldraw[RoyalPurple] (n4) circle (3pt) node[yshift=-.3cm,left] {control};
                        }

                        \onslide<3-> {
                            \draw[thick,ForestGreen,->,>=latex] (0,0) -- (10:1.5) node[midway,below] {$\mathbf{v_1}$};
                        }

                        \onslide<4-> {
                            \draw[thick,ForestGreen,->,>=latex] (0,0) -- (190:1.5) node[midway,above] {$\mathbf{v_2}$};
                        }

                        \filldraw (0,0) circle (1pt);
                    \end{tikzpicture}
                    \caption{Representation in $S^1$}
                \end{figure}
            \end{minipage}
        \end{frame}

    \section{Orthogonal constraint}
    
        \begin{frame}{Summary of orthogonal constraints}
            \begin{figure}
                \begin{subfigure}[t]{0.24\textwidth}
                    \centering
                    \href{run:frame_0_constraints.mp4?autostart&loop}{\includegraphics[width=\textwidth,height=\textwidth]{build/imgs/frame_0_constraints}}
                    \caption{No constraints}
                    \label{fig:constraints_0}
                \end{subfigure}
                \hfill
                \begin{subfigure}[t]{0.24\textwidth}
                    \centering
                    \href{run:frame_1_constraints.mp4?autostart&loop}{\includegraphics[width=\textwidth,height=\textwidth]{build/imgs/frame_1_constraints}}
                    \caption{1 constraint $(i_0 \bot k_1)$}
                    \label{fig:constraints_1}
                \end{subfigure}
                \hfill
                \begin{subfigure}[t]{0.24\textwidth}
                    \centering
                    \href{run:frame_2_constraints.mp4?autostart&loop}{\includegraphics[width=\textwidth,height=\textwidth]{build/imgs/frame_2_constraints}}
                    \caption{2 constraints $(i_0 \bot k_1)$, $(j_0 \bot k_1)$}
                    \label{fig:constraints_2}
                \end{subfigure}
                \hfill
                \begin{subfigure}[t]{0.24\textwidth}
                    \centering
                    \href{run:frame_3_constraints.mp4?autostart&loop}{\includegraphics[width=\textwidth,height=\textwidth]{build/imgs/frame_3_constraints}}
                    \caption{3 constraints $(i_0 \bot k_1)$, $(j_0 \bot k_1)$, $(j_0 \bot i_1)$}
                    \label{fig:constraints_3}
                \end{subfigure}
                \caption{All combinations of orthogonal constraints in 3 dimensions}
            \end{figure}
        \end{frame}




    % \section{Control}

    %     \begin{frame}{Control}
    %         \centering
    %         \begin{minipage}{0.7\textwidth}
    %             \begin{block}{Control inputs}
    %                 \centering
    %                 Controlled physical quantites are linear acceleration $\mathbf{a_r}$ and angular velocity $\mathbf{\omega_r}$
    %             \end{block}
    %             \begin{block}{State Equation}
    %                 \begin{eqnarray}
    %                     \left\{
    %                         \begin{array}{rcl}
    %                             \dot{\mathbf{p}} & = & \mathbf{R} \cdot \mathbf{v_r} \\
    %                             \dot{\mathbf{R}} & = & \mathbf{R} \cdot (\mathbf{\omega_r} \wedge) \\
    %                             \dot{\mathbf{v_r}} & = & \mathbf{R}^T \cdot g(\mathbf{p}) + \mathbf{a_r} - \mathbf{\omega_r} \wedge \mathbf{v_r}
    %                         \end{array}
    %                     \right.
    %                 \end{eqnarray}
    %             \end{block}
    %         \end{minipage}
    %     \end{frame}

    %     \begin{frame}{Control}
    %         \centering
    %         \begin{minipage}{0.7\textwidth}
    %             \begin{figure}
    %                 \begin{tikzpicture}
    %                     \shade[ball color = gray!40, opacity = 0.4] (0,0) circle (2cm);
    %                     \draw[thick] (0,0) circle (2cm);
    %                     \draw [thick] (-2,0) arc (180:360:2 and 0.6) coordinate[pos=0.25] (I) coordinate[pos=0.7] (R2);
    %                     \coordinate (R1) at (0,1.2);
    %                     \path[thick,ForestGreen,->,>=latex] (R1) edge[bend left] node[pos=0.65,right,ForestGreen] {$\omega_r$} (R2);
    %                     \node[ForestGreen] at (R1) {$\bullet$} node[ForestGreen] at (R1) [above] {$\mathbf{R}_u$};
    %                     \node[ForestGreen] at (R2) {$\bullet$} node[ForestGreen] at (R2) [below] {$\mathbf{R}_v$};
    %                     \node[red] at (I) {$\bullet$} node[red] at (I) [below] {$\mathbf{I}$};
    %                     \draw[dashed] (2,0) arc (0:180:2 and 0.6);
    %                 \end{tikzpicture}
    %                 \caption{$SO(3)$}
    %             \end{figure}
    %         \end{minipage}
    %     \end{frame}

    %     \begin{frame}{RotUV}
    %         \centering
    %         \begin{minipage}{0.3\textwidth}
    %             \begin{figure}
    %                 \begin{tikzpicture}
    %                     \shade[ball color = gray!40, opacity = 0.4] (0,0) circle (2cm);
    %                     \draw[thick] (0,0) circle (2cm);
    %                     \coordinate (R1) at (0.5,1.2);
    %                     \draw [thick] (-2,0) arc (180:360:2 and 0.6) coordinate[pos=0.25] (I) coordinate[pos=0.35] (R2) coordinate[pos=0.7] (R3);
    %                     \path[thick,ForestGreen,->,>=latex] (R1) edge[bend left=20] node[pos=0.65,right,ForestGreen] {$\omega_r$} (R3);
    %                     \draw[thick,RoyalPurple,->,>=latex] (0,0) -- node[midway,above left] {$\mathbf{u}$} (R1);
    %                     \draw[thick,RoyalPurple,->,>=latex] (0,0) -- node[midway,above] {$\mathbf{v}$} (R2);
    %                     \node[ForestGreen] at (R1) {$\bullet$} node[ForestGreen] at (R1) [above] {$\mathbf{R}_u$};
    %                     \node[ForestGreen] at (R2) {$\bullet$} node[ForestGreen] at (R3) [below] {$\mathbf{R}_u^v \cdot \mathbf{R}_u$};
    %                     \node[red] at (I) {$\bullet$} node[red] at (I) [below] {$I$};
    %                     \node[ForestGreen] at (R2) {$\bullet$} node[ForestGreen] at (R2) [below] {$\mathbf{R}_v$};
    %                     \draw[dashed] (2,0) arc (0:180:2 and 0.6);
    %                 \end{tikzpicture}
    %                 \caption{$SO(3)$}
    %             \end{figure}
    %         \end{minipage}
    %         \hfill
    %         \begin{minipage}{0.55\textwidth}
    %             \begin{block}{Vector to vector formulation}
    %                 \begin{equation}
    %                     \begin{array}{rcl}
    %                         \mathbf{K}_u^v & = & vu^T - uv^T \\
    %                         \mathbf{R}_u^v & = & \mathbf{I} + \mathbf{K}_u^v + \frac{1}{1 + \langle u, v\rangle} (\mathbf{K}_u^v)^2
    %                     \end{array}
    %                 \end{equation}
    %             \end{block}
    %         \end{minipage}
    %     \end{frame}

     % \begin{frame}{Vector to vector formulation}
        %     \begin{minipage}{0.4\textwidth}
        %         \centering
        %         \begin{figure}
        %             \begin{tikzpicture}
        %                 \shade[ball color = gray!40, opacity = 0.4] (0,0) circle (2cm);
        %                 \draw[thick] (0,0) circle (2cm);
        %                 \coordinate (R1) at (0.8,1.3);
        %                 \draw[thick,red,->,>=latex] (0,0) -- node[midway,above left] {$\mathbf{u}$} (R1);
        %                 \begin{scope}[rotate around={-30:(0,0)}]
        %                     \draw[thick] (-2,0) arc (180:360:2 and 0.6) coordinate[pos=0.2] (R3) coordinate[pos=0.36] (R4);
        %                     \draw[dashed] (2,0) arc (0:180:2 and 0.6);
        %                 \end{scope}
        %                 \draw[thick,RoyalBlue,->,>=latex] (0,0) -- node[midway,left,yshift=0.4cm] {$u_\bot$} (R3);
        %                 \onslide<2->{
        %                     \coordinate (R2) at (-0.2,1.15);
        %                     \draw[thick,ForestGreen,->,>=latex] (0,0) -- node[midway,above left] {$\mathbf{v}$} (R2); 
        %                 }
        %                 \onslide<3->{
        %                     \draw[thick,dotted,RoyalBlue,->,>=latex] (0,0) -- node[midway,below right] {$\mathbf{v_\bot}$} (R4);
        %                     \path[thick,->,>=latex,RoyalPurple] (R2) edge[in=70,out=210] (R4);
        %                     \coordinate (R5) at (0.4,-0.5);
        %                     \draw[thick,RoyalPurple,->,>=latex] (0,0) -- node[midway,above right] {$\mathbf{w}$} (R5);
        %                 }
        %             \end{tikzpicture}
        %             \caption{Representation in $S^2$}
        %         \end{figure}
        %     \end{minipage}
        %     \hfill
        %     \begin{minipage}{0.55\textwidth}
        %         \begin{block}<2->{Vector to vector formulation}
        %             \begin{equation}
        %                 \begin{array}{rcl}
        %                     \mathbf{K}_v^{u_\bot} & = & u_\bot v^T - vu_\bot^T \\
        %                     \mathbf{R}_v^{u_\bot} & = & \mathbf{I} + \mathbf{K}_v^{u_\bot} + \frac{1}{1 + \langle v, u_\bot\rangle} (\mathbf{K}_v^{u_\bot})^2 \\
        %                 \end{array}
        %             \end{equation}
        %         \end{block}
        %         \begin{block}<3->{Angular velocity w}
        %             \begin{equation}
        %                 \mathbf{w} = \wedge^{-1}(log(\mathbf{R}_v^{u_\bot}))
        %             \end{equation}
        %         \end{block}
        %     \end{minipage}
        % \end{frame}

    % \appendix

    % \begin{frame}[standout]
    %     Questions?
    % \end{frame}

    % \begin{frame}{Torpedo model}
    %     \centering
    %     \begin{minipage}{0.7\textwidth}
    %         \begin{block}{Torpedo assumptions}
    %             The following statements are equivalent
    %             \begin{itemize}
    %                 \item No side slip effect \\
    %                 \item No lateral speed
    %                 \item Velocity along the $\mathnormal{x}$ axis \\
    %             \end{itemize}
    %         \end{block}
    %         \begin{block}{Torpedo velocity}
    %             \begin{equation}
    %                 \mathbf{v_r} = (v_r, 0, 0)^T
    %             \end{equation}
    %         \end{block}
    %     \end{minipage}
    % \end{frame}

    % \begin{frame}{State Equation}
    %     \centering
    %     \begin{minipage}{0.7\textwidth}
    %         \begin{block}{AUV state}
    %             The state of the AUV is denoted by $\mathbf{X} = (\mathbf{p}, \mathbf{v_r}, \mathbf{R})$, where:
    %             \begin{itemize}
    %                 \item $\mathbf{p}$ is the position of the AUV \\
    %                 \item $\mathbf{v_r}$ is the velocity of the AUV in its frame \\
    %                 \item $\mathbf{R}$ is the rotation matrix between the world and the AUV \\
    %             \end{itemize}
    %         \end{block}
    %         \begin{block}{State Equation}
    %             \begin{eqnarray}
    %                 \left\{
    %                     \begin{array}{rcl}
    %                         \dot{\mathbf{p}} & = & \mathbf{R} \cdot \mathbf{v_r} \\
    %                         \dot{\mathbf{R}} & = & \mathbf{R} \cdot (\mathbf{\omega_r} \wedge) \\
    %                         \dot{\mathbf{v_r}} & = & \mathbf{R}^T \cdot g(\mathbf{p}) + \mathbf{a_r}
    %                     \end{array}
    %                 \right.
    %             \end{eqnarray}
    %         \end{block}
    %     \end{minipage}
    % \end{frame}

    % \begin{frame}{Inputs}
    %     \centering
    %     \begin{minipage}{0.7\textwidth}
    %         \begin{block}{Inputs}
    %             \centering
    %             Input vector of the system is $\mathbf{u} = (u_0, u_1, u_2, u_3)^T$, where:
    %             \begin{itemize}
    %                 \item $u_0$: thruster velocity \\
    %                 \item $u_1, u_2, u_3$: fin angles
    %             \end{itemize}
    %         \end{block}
    %     \end{minipage}
    % \end{frame}

    % \begin{frame}{Linear acceleration}
    %     \centering
    %     \begin{minipage}{0.7\textwidth}
    %         \begin{block}{Assumptions}
    %             \vspace{0.2cm}
    %             \begin{itemize}
    %                 \item Quadratic thrust with velocity
    %                 \item Quadratic drag with velocity
    %             \end{itemize}
    %         \end{block}
    %         \begin{block}{Linear acceleration}
    %             \begin{equation}
    %                 \mathbf{a_r} = \underbrace{p_1 \cdot \left(\begin{smallmatrix}u_0 \\ 0 \\ 0 \end{smallmatrix}\right)^2}_{thrust} - \underbrace{p_2 \cdot \mathbf{v_r} \cdot |\mathbf{v_r}|}_{drag}
    %             \end{equation}
    %         \end{block}
    %     \end{minipage}
    % \end{frame}

    % \begin{frame}{Angular velocity}
    %     \centering
    %     \begin{minipage}{0.8\textwidth}
    %         \begin{block}{Assumptions}
    %             \vspace{0.2cm}
    %             \begin{itemize}
    %                 \item Fin's drag negligible compared to the AUV drag
    %                 \item Fin's lift used to control the AUV
    %                 \item Direct response between the fin's angle and the angular velocity
    %             \end{itemize}
    %         \end{block}
    %         \begin{block}{Angular velocity}
    %             \begin{equation}
    %                 \mathbf{\omega_r} = v_r^{\color{red}{2}} \cdot 
    %                     \underbrace{
    %                         \left(
    %                         \begin{smallmatrix}
    %                             -p_3 & -p_3 & -p_3 \\
    %                             0 & p_4 \cdot sin(\frac{2\pi}{3}) & -p_4 \cdot sin(\frac{2\pi}{3}) \\
    %                             p_4 & p_4 \cdot cos(\frac{2\pi}{3}) & p_4 \cdot cos(\frac{2\pi}{3})
    %                         \end{smallmatrix}
    %                         \right)
    %                     }_{\mathbf{B}(p_3, p_4)} \cdot \left(\begin{smallmatrix}u_1\\ u_2\\ u_3\end{smallmatrix}\right)
    %             \end{equation}
    %         \end{block}
    %     \end{minipage}
    % \end{frame}

    % \begin{frame}{Repartition matrix}
    %     \centering
    %     \begin{minipage}{0.8\textwidth}
    %         \begin{block}{Fin's lift}
    %             $\forall i \in \{0, 1, 2\}:$
    %             \begin{itemize}
    %                 \item Force $\mathbf{f_i} = \alpha \cdot u_i \cdot v^2 \cdot (0, 1, 0)^T$
    %                 \item Orientation $\mathbf{R_i} = \mathbf{R_x}\left(\frac{2i\pi}{3}\right)$
    %                 \item Center of pressure $\mathbf{q_i} = (\mathit{l_x}, 0, \mathit{l_z})^T$
    %                 \item Torque $\mathbf{\tau_i} = \mathbf{R_i} \cdot \mathbf{q} \wedge \mathbf{f_i}$
    %             \end{itemize}
    %         \end{block}
    %         \begin{block}{Angular velocity}
    %             \begin{equation}
    %                 \mathbf{\omega_r} = v_r^2 \cdot 
    %                     \underbrace{
    %                         \left(
    %                         \begin{smallmatrix}
    %                             -p_3 & -p_3 & -p_3 \\
    %                             0 & p_4 \cdot sin(\frac{2\pi}{3}) & -p_4 \cdot sin(\frac{2\pi}{3}) \\
    %                             p_4 & p_4 \cdot cos(\frac{2\pi}{3}) & p_4 \cdot cos(\frac{2\pi}{3})
    %                         \end{smallmatrix}
    %                         \right)
    %                     }_{\mathbf{B}(p_3, p_4)} \cdot \left(\begin{smallmatrix}u_1\\ u_2\\ u_3\end{smallmatrix}\right)
    %             \end{equation}
    %         \end{block}
    %     \end{minipage}
    % \end{frame}
    
\end{document}