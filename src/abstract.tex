\documentclass[14pt, a4paper]{article}

\usepackage{extsizes}
\usepackage{amsmath}
\usepackage{amsthm}
\usepackage{amssymb}
\usepackage{url}
\usepackage[utf8]{inputenc}
\pagenumbering{gobble}
\clearpage

% USER PACKAGES
\usepackage{gensymb}
\usepackage[backend=biber,style=ieee,doi=false,isbn=false,url=false,eprint=false]{biblatex}
\addbibresource{bib/abstract.bib}

\begin{document}

	\begin{center}

		{\Large\bf Torpedo-like AUV control in constrained environment}

		\vspace*{0.8cm}

		{\large Quentin \textsc{Brateau}$^{1}$, Luc \textsc{Jaulin}$^{1}$}

		\bigskip

		{\small $^{1}$ENSTA Bretagne, UMR 6285, Lab-STICC, \\
		2 rue François Verny, 29806 Brest CEDEX 09, \textsc{France} \\
		\texttt{quentin.brateau@ensta-bretagne.org}
		}

	\end{center}

	\bigskip

	{\noindent\bf Keywords:} AUV, Control, Simulation, Inertial

	AUVs have many advantages compared to ROVs, such as the absence of a tether and the possibility of carrying out missions covering large areas. On the other hand, AUVs must have autonomy in terms of decision-making and energy. It is therefore necessary to design robust control laws to make the navigation of such vehicles safer. A model of a torpedo AUV will be proposed for simulation purposes and the development of control laws. Then, simple control laws will be presented for inertial navigation in a constrained environment, such as a swimming pool or a harbor.

	\medskip

\end{document}
